\documentclass[output=paper]{langsci/langscibook}
\ChapterDOI{10.5281/zenodo.573784}

\author{Larry M. Hyman\affiliation{University of California, Berkeley}}
\title{What (else) depends on phonology?} 
\epigram{To construct phonology so that it mimics syntax is to miss a major result of the work of the last twenty years, namely, that syntax and phonology are essentially different.}
\epigramsource{\citep[69]{Bromberger1989}}
\maketitle
\begin{document}
 


\section{Is phonology different?}

\is{phonology} \is{typology} In \citet{Hyman2007} I asked, “Where’s phonology in typology?” While phonology turned out to be well represented at the Ardennes workshop and this volume of proceedings, it is typically underrepresented, even ignored by some typologists. I considered three reasons:

\begin{itemize}
 \item[(i)] Phonology is different (cf. the above Bromberger \& Halle quote).
 \item[(ii)] Phonological typology may seem uninteresting to typologists, particularly if defined as follows:

\begin{quote}

 “[. . . ] it is possible to classify languages according to the phonemes they contain.... Typology is the study of structural features across languages. Phonological typology involves comparing languages according to the number or type of sounds they contain.” \citep{Vajda2001}
\end{quote}

 \item[(iii)]   Phonology is disconnected from the rest (e.g. from morphosyntactic typology).
\end{itemize}

\is{phonology} \is{typology} As evidence that phonology is underrepresented, I noted that there is no coverage in \citegen{Whaley1997} textbook, \textit{Introduction to Typology}. The more recent \textit{Oxford Handbook of Linguistic Typology} \citep{Song2011} provides confirmation of the above assessment:

\begin{itemize}
 \item[(i)] Phonology is \textit{underrepresented:} there is only one chapter on phonology out of thirty (= 1/30) constitituing 13 out of 665 pages (= 2\%)
 \item[(ii)] Phonology is seen as\textit{ different: }Why isn’t Chapter 24 entitled “Phonological Typology”, parallel with the other chapters?
\\
    Chapter 21:  Syntactic typology (Lindsay Whaley)
\\
    Chapter 22:  Morphological typology (Dunstan Brown)
\\
    Chapter 23:  Semantic typology (Nicholas Evans)
\\
    BUT:  Chapter 24: Typology of phonological systems (Ian Maddieson)
\item[(iii)] Phonology is \textit{ignored:} There is no mention of phonology in Chapter 10 “Implicational Hierarchies” (Greville Corbett), which has sections on syntactic (§3.1), morphosyntactic (§3.2) and lexical (§3.3) hierarchies. As a phonological example the chapter could easily have cited and illustrated the sonority hierarchy \citep{Clements1990} and the claim that if a lower sonority segment can function as the nucleus of a syllable, then a higher sonority segment in a column to its right also can; see Table \ref{tab:hyman:hierarchy}.
\end{itemize}

\begin{table}
\fittable{
\begin{tabular}{cp{0cm}cp{0cm}cp{0cm}cp{0cm}cll}
\lsptoprule
 Obstruent & {\textless} & Nasal & {\textless} & Liquid & {\textless} & Glide & {\textless} & Vowel \\
 \midrule
 {}- &  & {}- &  & {}- &  & {}- &  & + &  & syllabic  \\
 {}- &  & {}- &  & {}- &  & + &  & + &  & vocoid \\
 {}- &  & {}- &  & + &  & + &  & + &  & approximant  \\
 {}- &  & +   &  & + &  & + &  & + &  & sonorant  \\
 0   &  & 1   &  & 2 &  & 3 &  & 4 &  & \textit{rank}   (\textit{degree of sonority}) \\
\lspbottomrule
\end{tabular}
}
\caption{The sonority hierarchy: An implicational hierarchy in phonological typology}
\label{tab:hyman:hierarchy}
\end{table}

  There are of course exceptions to the above: \textit{WALS Online} \citep{DryerEtAl2013} includes 19 chapters on phonology out of 144 (or 13.2\%). There also are several phonological databases and typological projects which are concerned with how phonology interfaces with the rest of grammar, e.g. \citet{BickelEtAl2009}, based on the Autotyp project \citep{BickelEtAl2016}. Still, phonology is at best incidental or an afterthought in much of typological work. This stands in marked contrast with the work of Joseph Greenberg, the father of modern linguistic typology, whose foundational work on typology and universals touched on virtually all aspects of phonology, e.g. syllable structure \citep{Greenberg1962,Greenberg1978a}, distinctive features \citep{Greenberg1967}, vowel harmony \citep{Greenberg1963}, nasalized vowels  \citep{Greenberg1966phonology,Greenberg1978b}, glottalized consonants \citep{Greenberg1970}, word prosody \citep{Greenberg1976}. Note also that one full volume out of the four volumes of \citet{Greenberg1978Universals} was dedicated to phonology!

  There are at least two reasons why phonological typology, properly conducted, can be relevant to scholars outside of phonology. First, there are lessons to be learned that are clearest in phonology, e.g. concerning dependencies, the central  issue of this volume. Second, there have been claims that grammatical typology can be dependent on phonology. I take these both up in the following two sections.

\section{Dependencies require analysis (which requires theory)}

It is interesting that Greenberg typically cited phonological examples to make the didactic point that any property found in a language can be stated as an implicans on an absolute universal implicatum:
\is{universals} \is{implicational universals} \is{oral vowels} \is{nasal vowels}
\begin{quote}
We have the unrestricted universal that all languages have oral vowels and the implicational universal that the presence of nasal vowels in a language implies the presence of oral vowels, but not vice-versa. \citep[509]{Greenberg1966phonology}
\end{quote}

\begin{quote}
Of course, where an unrestricted universal holds, any statement may figure as implicans. For example, if a language has a case system, it has oral vowels. \citep[509]{Greenberg1966phonology}
\end{quote}


However, phonology teaches us two additional lessons: (i) Dependencies are themselves highly dependent on the level of analysis. (ii) The analysis however varies according to the theory adopted. To illustrate the first point, let us stay with the example of nasality which, in different languages, may be underlyingly contrastive (\tabref{tab:hyman:nasality}).

\begin{table}
\begin{tabular}{llllll}
\lsptoprule
(i) & on consonants only: & /m, n, ŋ/ &  & e.g.  \ili{Iban} & \\
(ii) & on vowels and consonants: & /ĩ, ũ, ã, m, n, ŋ/ &  & e.g.  \ili{Bambara} & \\
(iii) & on vowels only: & /ĩ, ũ, ã/ &  & e.g.  \ili{Ebrié} & \\
(iv) & on whole morphemes: & /CVC/\textsuperscript{N} &  & e.g.  \ili{Barasana} & \\
(v) & absent entirely: & {}-{}-{}-{}-{}- &  & e.g.  \ili{Doutai} & \\
\lspbottomrule
\end{tabular}
      
    \caption{A typology of nasal contrasts (cf. \citealt{Cohn1993,Clements2005})}
    \label{tab:hyman:nasality}
\end{table}

      
\is{nasal consonants} A problem arises when we attempt to typologize on the basis of languages which have vs. do not have underlying nasal consonants. The class of languages lacking underlying nasal consonants is not coherent, as this includes three different situations: languages like \ili{Ebrié} (iii) which contrast nasality only on vowels; languages like \ili{Barasana} (iv) which have nasal prosodies, e.g. /bada/\textsuperscript{N}  [mãnã]; languages like \ili{Doutai} (v) which lack nasality altogether.

While (v) represents an observable (“measurable”) fact, assuming that there is also no nasality on the surface, (iii) and (iv) represent linguistic analyses designed to factor out the surface nasality by assigning the oral/nasal contrast either to vowels or to whole morphemes -- ignoring the fact that these language have output nasal consonants. To appreciate the fact that languages with contrastive nasality on vowels only (iii) always have surface phonetic nasal consonants, consider the case of \ili{Ebrié}, a \ili{Kwa} language of Ivory Coast:

\begin{quote}
 ... nous considérons que l’ébrié ne possède aucune consonne nasale phonologique et que [m], [n] et [ɲ] sont les allophones respectifs de /ɓ/, /ɗ/ et /y/ [before nasalized vowels] \citep[25]{Dumestre1970} 
\end{quote}


In this language, /ɓa, ɗa, ya/ are realized [ɓa, ɗa, ya],while /ɓã, ɗã, yã/ are realized [mã, nã, ɲã]. This analysis is possible because there are no sequences of *[ɓã, ɗã, yã] or *[ma, na, ɲa]. Since contrasts such as /ta/ vs. /tã/ independently require a [+nasal] specification on vowels, the structure-sensitive phonologist cannot resist generalizing: only vowels carry an underlying [+nasal] specification to which a preceding /ɓ, ɗ, y/ assimilate.

\is{surface properties} The \ili{Ebrié} example neatly illustrates the fact that there is no language which has \textsc{surface} nasality only on vowels. This raises the question of what level of representation is appropriate for typological purposes: underlying (phonemic) or surface (allophonic)? While \citet[24]{Hockett1963} once noted that “phonemes are not fruitful universals,” since they are subject to the individual linguist’s interpretation of “the facts”, the question is whether the same applies to typological generalizations. As I like to put it, we aim to typologize the linguistic properties, not the linguists. At the Ardennes workshop Martin Haspelmath argued forcefully that observable “surface” properties are the facts and that they should serve as input to typology. If so, we must then address the question of what to do about vowel nasalization in \ili{English}. As often pointed out, a word like \textit{can’t} is often pronounced [k{\~{æ}}nt] or even [k{\~{æ}}t], in contrast with \textit{cat} [kæt]. The usual assumption is that such variations should be attributed to phonetic implementation \citep{Cohn1993}, i.e. a third level. While this raises the possibility of a different kind of typology based on surface phonetic contrasts, however they may be obtained, thereby blurring the difference between phonetics and phonology, I argue instead for a phonological typology based more strictly on a more structural level of representation. \ili{English} thereby falls into category (i) in the above typology.\footnote{ As this volume was going to press I received \citet{KiparskyToAppear} which also addresses this question. Concerned with universals and UG, Kiparsky proposes that phonological typology should not be based on the phonemic level, rather what he terms the “lexical level” which contains salient redundancies. At this level \ili{Ebrié} would have a nasal contrast on both consonants and vowels thereby allowing the universalist to claim that a language which contrasts nasalized vowels also has nasal consonants.}

A related question is how we should state the dependency. In an earlier paper I tried to capture the dependency by referring to both levels:

\ea
Vocalic Universal \#6: A vowel system can be contrastive for nasality only if there are output nasal consonants [i.e. surface phonetic nasal consonants] \citep[99]{Hyman2008}
\z


\is{nasal consonants} To rephrase this: If a vowel system is underlyingly contrastive for nasality, there will always be output nasal consonants, as in \ili{Ebrié}. However, it appears that this is not general enough: the underlying nasality on vowels may be irrelevant, given systems with prosodic nasality such as \ili{Barasana}. An alternative is: \is{prosodic nasality} 
\ea
Consonantal Universal: A phonological system can be contrastive for nasality only if there are output nasal consonants (i.e. independent of whether the consonant nasality is underlying or derived, and whether nasality is underlyingly segmental or prosodic)
\z

This is true of all four of the systems (i)-(iv) which have contrastive nasality. Thus, the implicans can be either the underlying vowel system or the whole phonological system. We thus are able to relate the dependencies about observable “facts” with our (interesting) analyses of them. The same point can be made concerning vertical vowel systems: Systems such as \ili{Kabardian} or \ili{Marshallese} are often analyzed as /ɨ, ə, a/, /ɨ, a/ etc., but always have output [i] and [u] (cf. Vocalic Universal \#5 in Hyman 2008: 98).

\is{oral vowels} \is{implosives} Above I cited Greenberg’s absolute universal “all languages have oral vowels” as a universally available implicatum (“if a language has a case system, it has oral vowels”). What about an implicans that is extremely rare? The velar implosive [ɠ] is very rare in languages:

\begin{quote}
The velar implosive is a very infrequent sound and... always seems to imply the presence of bilabial, apical, and palatal members of the series. \citep[128]{Greenberg1970}
\end{quote}


What then can be predicted from its presence? Note first that implosives occur in 53 out of the 451 languages in the UPSID database \citep{Maddieson1990}. A bilabial implosive occurs in 50 of these 53 languages, while an apical (dental or alveolar) implosive occurs in 42 languages. In stark contrast, a velar implosive occurs in only five of the 53 languages. In \tabref{tab:hyman:implosives} I attempt to establish dependencies “if ɠ, then X” again to determine the role of analysis in establishing implicational universals.

\begin{table}
\begin{tabularx}{\textwidth}{p{2.6cm}Xccccc}
\lsptoprule
&  & { \textit{Chadic}} & { \textit{Omotic}} & \multicolumn{3}{c}{ \textit{East Sudanic}}\\
&  & { Tera} & { Hamer} & { Ik} & { Maasai} & {Nyangi}\\
\midrule
% \hhline{~~-----}
\raggedright
other implosive consonants: & if /ɠ/, \newline then /ɓ, ɗ/ & ${\surd}$ & ${\surd}$ & ${\surd}$ & ${\surd}$ & ${\surd}$\\
\\	
% \hhline{~~-----}
\raggedright
basic voiceless consonants: & if /ɠ/, \newline \mbox{then /p, t, k/} & ${\surd}$ & ${\surd}$ & ${\surd}$ & ${\surd}$ & ${\surd}$\\
\\
% \hhline{~~-----}
\raggedright
voiced non-implosives? & if /ɠ/, \newline \mbox{then /b, d, g/} & ${\surd}$ & ${\surd}$ & ${\surd}$ & * & *\\
% \hhline{~~-----}
\lspbottomrule
\end{tabularx}
\caption{Possible implicational university based on the presence of contrastive /ɠ/}
\label{tab:hyman:implosives}
\end{table}

As seen, if a language has /ɠ/ we can predict that the other two implosives will be present, as well as voiceless stops. While \ili{Maasai} and \ili{Nyangi} appear to falsify the implication “if ɓ, ɗ, ɠ, then b, d, g”, it can be saved if we re-analyze [ɓ, ɗ, ɠ]  as /b, d, g/, which are lacking in the two systems. I would argue against this as a valid move, but it again underscores the problem of level of analysis, which provides us with two different kinds of claims:

\begin{itemize}
 \item[(i)]  a descriptive claim:  if a language has [ɓ, ɗ, ɠ], it will have contrastive /b, d, g/
 \item[(ii)]  an analytic claim: if a language has [ɓ, ɗ, ɠ] it will have /b, d, g/ (either contrastively or not)
\end{itemize}

  

  

The above summarizes a bit of what we face in phonology. What about grammar depending on phonology?

\section{Non-arbitrary ≠ predictive}

\is{absolute universals} In this section I begin by considering the empirical bases in establishing a dependency. Specific implicans-implicatum of dependencies are arrived at in a number of ways, combining degrees of inductive observation and deductive reasoning. In this section I consider two types of dependencies which appear to be “non-arbitrary”: (i) those which depend on (claimed) absolute universals; (ii) those which depend on historically linked events. To begin with the first, ultimately false claims may at first appear to be based on what the proposer considers to have an external (e.g. physical phonetic) basis:

\begin{quote}
  “Since sequences containing only pure consonants, such as [kpt\v{c}sm] or [rʃtlks], cannot be pronounced, all words must include at least one vowel or vowel-like (vocalic, syllabic) sound segment”, 
  
  hence: 
  
  “In all languages, all words must include at least one vocalic segment.”\linebreak \citep[153]{Moravcsik2013}
\end{quote}

This statement contains the dependency, “If X is a word, then it contains at least one vocalic segment,” which however is false, as seen in the following \ili{Bella Coola} voiceless obstruent utterance (\citealt[5]{Nater1984}, cited by \citealt[1]{Shaw2002}):

\ea
  xɬp’⁠χ⁠ʷɬtɬpɬɬs  kʷc’ 
  \glt ‘then he had had in his possession a bunchberry plant’
  \z


In this case there was an extra-linguistic basis to the claim--languages can’t have words that are universally unpronounceable. On the other hand, linguists have been known to make arbitrary “universal stabs in the dark” which have no obvious linguistic or extra-linguistic basis, e.g. “No language uses tone to mark case” (Presidential Address, 2004 Annual Linguistic Society of America Meeting, Boston). Stated as a dependency: %%does this need a reference? None found in bibliography
% \todo[inline]{@Author: Nick would like a reference for this Presidential address to go in the bibliography}
\begin{itemize}
 \item[(i)] If a language has tone, it will not be used to mark case.
 \item[(ii)]  If a language has case, it won’t be marked by tone.
\end{itemize}

But consider Table \ref{tab:hyman:maasai} from \ili{Maasai} \citep[177--184]{Tucker1955}, where the acute (\'{}) marks H(igh) tone, while the grave (\`{}) accent marks L(ow) tone:
% \todo[inline]{Check reference: is lowercase \textbf{tompo} Ole Mpaayei correct?}
\begin{table}
\begin{tabularx}{\textwidth}{lXXlX} 
\lsptoprule
& {nominative} & {accusative}   & & {{nom. vs. acc. \newline \mbox{tone patterns}}}\\
\midrule
%\multicolumn{5}{l}{\todo[inline]{èlʊkʊnyá seems identical}}\\
class I: & èlʊ̀kʊ̀nyá & èlʊ́kʊ́nyá  & ‘head’ & L\textsuperscript{n}{}-H vs. L-H\textsuperscript{n}\\
& èncʊ̀màtá & èncʊ́mátá & ‘horse’ & \\
\\
class II: & èndérònì & èndèrónì & ‘rat’ & H on ${\sigma}$\textsubscript{1} vs. ${\sigma}$\textsubscript{2}\\
& ènkólòpà & ènkòlópà & ‘centipede’ & \\
\\
class III: & òlmérégèsh & òlmérègèsh & ‘ram’ & \multirow{2}{*}{\parbox{3cm}{H on ${\sigma}$\textsubscript{2} \& ${\sigma}$\textsubscript{3} vs. \newline on ${\sigma}$\textsubscript{2} only}}\\
& òlósówùàn & òlósòwùàn & ‘buffalo’ & \\
\\
class IV: & òmótònyî & òmótònyî & ‘bird’ & identical tones\\
& òsínkìrrî & òsínkìrrî & ‘fish’ & \\
\lspbottomrule
\end{tabularx}
\caption{Case marking by tone in Maasai}
\label{tab:hyman:maasai}
\end{table}

In reality, if tone can be a morpheme (which is uncontroversial), it can do anything that a morpheme can do! What innate or functional principle would block tone from marking case? \is{functional principles}

\is{grammar-phonology dependencies} The above examples reveal a temptation to claim a non-arbitrary relation between certain aspects of grammar and phonology. Recently there has been renewed interest in pursuing a centuries-old “intuition” that certain aspects of syntax and morphology are not only interdependent, but also dependent on phonology. The standard reference is \citet{Plank1998}, who attributes the following positions to:

\begin{quote}
  \textit{Encyclopaedia Brittannica} (1771): “Words tend to be longer than one syllable in transpositive [free word order] languages and to be monosyllabic in analogous [rigid word order] languages." \citep[198]{Plank1998}
  
  W. \citet{Radioff1882}: “(a) If vowel assimilation is progressive (= vowel harmony), then the morphology will be agglutinative (and indeed suffixing), but not vice versa.... (b) if the morphology is flective, then if there are vowel assimilations they will be regressive (= umlaut), but not vice-versa....” \citep[202]{Plank1998}

  \newpage 
  Rev. James \citet{Byrne1885}: “Unlimited consonant clustering correlates with VS order, limitations on consonant clustering correlate with SV order.” \citep[200]{Plank1998}

  Georg von der \citet{Gabelentz1901}: Languages with anticipatory phonological assimilation should have anticipatory grammatical agreement (e.g. from N to A in an A-N order), while languages with perseverative phonological assimilation should have perseverative grammatical agreement (e.g. from N to A in an N-A order). (my paraphrasing of \citealt{Plank1998}: 197); also \citet{Bally1944}: Séquence Progressive vs. Séquence Anticipatrice \citep[211]{Plank1998}
%'my paraphrasing' refers to Hyman 
\end{quote}


Interestingly, Greenberg did not buy into this. Grammar does appear in examples involving the universality of oral vowels, which was didactically exploited as an implicatum to show that any arbitrary implicans follows -- grammatical ones are typically cited \citep{Greenberg1966phonology,Greenberg1978b}:
\begin{itemize}
 \item[(i)]  If a language has case, it also has oral vowels (repeated from above)
 \item[(ii)]  If a language has sex-based gender, it also has oral vowels
 \item[(iii)]  If a language doesn’t have oral vowels, the language doesn’t have sex-based gender (or maybe it does)
\end{itemize}
  

\is{oral vowels} What this reveals is that there is a world of difference between correlation and causation. Noone would ever claim that the presence of oral vowels has something to do with any of the above grammatical properties. As \citet{Plank1998} put it:

\begin{quote}
  “Although these implications all happen to be true, their typological value is nil.” \citep[223]{Plank1998}
\end{quote}


The last century has seen a proliferation of proposals to distinguish language “types” which identify various phonological properties with grammatical ones, either as non-directional correlations (P$\leftrightarrow $G) or with one dependent on the other (P${\rightarrow}$G, G${\rightarrow}$P), e.g.

\begin{itemize}
 \item anticipatory vs. progressive languages
 \item iambic vs. trochaic languages
 \item stress-timed vs. syllable-timed vs. mora-timed languages
 \item syllable vs. word languages
 \item word vs. phrase languages  
\end{itemize}

(See especially proposals of Bally, Skalička, Lehmann, Dressler, Donegan \& Stampe, Dauer, Gil, Auer, all in \citealt{Plank1998}.)
As an example, consider the following two languages types from \citeauthor{Lehmann1973} (1973 et seq), as summarized by \citet[208]{Plank1998} (\tabref{tab:hyman:think}).

\begin{table}
\begin{tabularx}{\textwidth}{XX}
\lsptoprule
\textit{“think \ili{Turkish} or \ili{Japanese}”} & \textit{“think \ili{Germanic}”}\\
\midrule
• dependent-head (OV, AN etc.) & • head-dependent (VO, NA etc.)\\
• suffixes & • prefixes\\
• agglutination (exponents = loosely bound affixes) & • flection (exponents = tightly fused with stem)\\
• no agreement & • agreement\\
• vowel harmony (progressive, root triggers) & • umlaut (= regressive, suffix triggers)\\
• few morphophonological rules (mostly progressive) & • many morphophonological rules (mostly regressive)\\
• syllable structure simple & • syllable structure complex\\
• pitch accent & • stress accent + unstressed vowel reduction\\
• mora-counting & • syllable-counting\\
\lspbottomrule
\end{tabularx}
\caption{Lehmann's Holistic Typology of Languages}
\label{tab:hyman:think}
\end{table}

While such grammar-phonology dependencies have not generally caught on in typological or in phonological circles, there is renewed interest in statistical correlations between phonological properties and OV vs. VO syntax \citep{NesporEtAl2011,Tokizaki2010,Tokizaki2012} (cf. \citealt{Cinque1993}) as well as word class, e.g. noun vs. verb, transitive vs. intransitive verbs \citep{Smith2011,Dingemanse2015,Fullwood2014}.
\begin{table}
\begin{tabular}{llllll} 
\lsptoprule
& \multicolumn{2}{l}{ {trochaic}} &  & \multicolumn{2}{l}{ {iambic}}\\
\midrule
obligatorily transitive & 506 & (39\%) &  & 804 & (61\%)\\
ambitransitive & 357 & (55\%) &  & 293 & (45\%)\\
obligatorily intransitive & 227 & (64\%) &  & 130 & (36\%)\\
\lspbottomrule
\end{tabular}
\caption{Stress Placement on Verbs in English}
\label{tab:hyman:fulwood}
\end{table}


Concerning the latter, Fullwood  demonstrates a statistical correlation between verb transitivity and stress on \ili{English} bisyllabic verbs (\tabref{tab:hyman:fulwood}). \is{stress} Although the absolute number of verbs having one vs. the other stress patterns is reasonably close (1090 trochaic, 1227 iambic), the smallest group by far are obligatorily intransitive iambic verbs such as \textit{desíst}. Here we can see the consequence of stress to avoid final position--and to especially avoid the “weak” utterance-final position where declarative intonation would normally realize a high to low falling pitch \citep[45]{Hyman1977}. Being utterance-internal is quite different. As \citet{Fullwood2014} puts it:

\begin{quote}
Words that frequently occur phrase-finally are more likely to retract stress from their final syllable, while other words that rarely occur in phrase-final position are quite happy to accommodate a final stress. \citep[130]{Fullwood2014}
\end{quote}


Similar proposals have been offered of a relation between word order and stress, but one of causation has not been widely accepted, whether based on universal tendencies or historically linked events.

\is{serial verb constructions} \is{African languages} \is{Southeast Asian languages} A case of the latter does comes from \citet[50-51]{Foley1985}, who offer “an interesting list of shared properties”, some phonological, some grammatical, among languages with valence-increasing serial verbs, particularly in West Africa and Southeast Asia:

\begin{itemize}
 \item [(i)]  phonemic tone  
 \item [(ii)]  many monosyllabic words   
 \item [(iii)]  isolating morphological typology 
 \item [(iv)]  verb medial word order (SVO)
\end{itemize}


They go on to explain:
\begin{quote}
This cluster of properties is not accidental: they are all interrelated. Phonological attribution causes syncope of segments or syllables, with the result that phonemic tone or complex vowel systems develop to compensate for phonemic distinctions being lost. On the grammatical side, phonological attrition causes gradual loss of the bound morphemes.... At this verbal morphology is lost, a new device for valence adjustment must be found. Verb serialization begins to be used in this function, \textit{provided serial constructions already exist in the language}. \citep[51]{Foley1985} [my emphasis]
\end{quote}


\is{serial verb constructions} \citeauthor{Foley1985} suggest that the development of serial verbs proceeds in the following order:

\ea  motion/directional verbs {\textgreater} postural verbs {\textgreater} stative/process verbs {\textgreater} valence
\z

Crucially, it is only the last (valence) stage that correlates with the above properties (vs. \citeauthor{Crowley2002} \citeyear{Crowley2002} re \ili{Oceanic} serial verbs which do not meet these criteria). It is the loss of head-marking on verbs (benefactive, instrumental applicatives etc), which was due to the introduction of prosodic size conditions on verb stems in NW \ili{Bantu} \citep{Hyman2004}, that feeds into verb serialization. Thus there is a \textit{non-arbitrary} relation between the phonological development, the loss of head-marking morphology, and the extended development of an analytical structure with serial verbs.

\is{prediction} \is{synchrony} \is{diachrony}  However, the cause-and-effect is not \textit{predictive}: Neither the synchronic nor diachronic interpretation of these dependencies holds true for all cases:

\begin{itemize}
  \item synchronic dependency:  if valence-marking serial verbs, then tone, tendency towards monosyllabicity, isolating morphology, SVO (but \ili{Ijo} = SOV)

  \item   diachronic dependency:  if serial verbs + phonological attrition, then valence-marking serial verbs, tone etc. (but some serial verb languages do not employ serial verbs to mark valence)

\end{itemize}

  

The diachronic alternative for marking benefactives, instruments etc. is with adpositions. \ili{Nzadi} is a Narrow \ili{Bantu} language spoken in the Democratic Republic of Congo which has broken down the \ili{Bantu} agglutinative structure to become analytic and largely monosyllabic. Serial verbs have not been introduced to replace lost verbal suffixes \citep{Crane2011}: 

\ea
\ea
  \gll  bɔ    ó   túŋ   ndzɔ  sám %ꜜ
  \texttt{ꜜ\hspace*{-.2em}}é  báàr  \\
    they \textsc{past} build  house reason of people \\
    \glt ‘they built a house for the people’  
\ex
    \gll ndé  ó  wɛɛ  \'{m}bùm  tí  ntáp  òté  \\
    he   \textsc{past}   pick    fruit   with branch tree\\
    \glt ‘he picked fruit with a stick’
\z
\z

The serial structures `*they built house give people' and `*he take stick pick fruit' are not used in \ili{Nzadi}, which is spoken outside the West African serial verb zone.

“Holistic” typologies such as the one from Lehmann presented above are still only “hopeful” \citep{Plank1998}, based to a large extent on the feeling that clustering of properties across phonology, morphology and syntax is non-arbitrary (e.g. \ili{Indo-European} and \ili{Semitic} vs. \ili{Uralic} and \ili{Altaic}; West Africa and Southeast Asia vs. \ili{Athabaskan}, \ili{Bantu}). But whatever links one can find between the cited properties, these effects are non-predictive. Still, linguists hold strong feelings on such interdependencies, and I’m guilty too. Thus, as my own observation (hope) I offer the following as a concluding proposal.

\is{agglutination} The highly agglutinative \ili{Bantu} languages contrast only two tone heights, H and L (often analyzed as privative /H/ vs. Ø). A third M(id) tone height is only present in languages which have broken down the morphology (thereby creating more tonal contrasts on the remaining tone-bearing units). Thus compare the H vs. L agglutinative structure in the \ili{Luganda} utterance in (\ref{ex:hyman:LHM}a) with the H vs. M vs. L isolating structure in (\ref{ex:hyman:LHM}b) of Fe’fe’-Bamileke, a Grassfields \ili{Bantu} language of Cameroon:

\ea\label{ex:hyman:LHM}
  \ea  
  \ili{Luganda}\\
  \gll à-bá-tá-lí-kí-gúl-ír-àgàn-à  \\
    \textsc{aug}{}-they-\textsc{neg-fut}{}-it-buy-\textsc{appl-recip-fv}  \\
    \glt ‘they who will not buy it for each other’
    (\textsc{aug} = augment; \textsc{fv} = inflectional final vowel)

  \ex  
  \ili{Fe’fe’-Bamileke}\\
  \gll à   kɑ̀   láh  pìɛ   náh ncw\={e}e mbɒ̀ɒ̀  hɑ̄  m\={u}\={u} \\
      he \textsc{past} take knife take   cut       meat   give  child \\
  \glt    ‘he cut the meat with a knife for the child’
       (   \={ }  = Mid tone)

\z
\z

The morphological structure of words in polyagglutinative languages like \ili{Luganda} is highly syntagmatic. This is most compatible with a tone system with privative /H/ vs. Ø, where the Hs are assigned to specific positions. (Although they don’t have a M tone, some \ili{Bantu} languages allow \texttt{ꜜ\hspace*{-.2em}}H, as tonal downstep is also syntagmatic.) A full contrast of /H, M, L/ on every tone-bearing unit would produce a huge number of tone patterns (3 x 3 x 3 etc.), so one should at best expect the /H, M, L/ contrast to occur only on prominent positions (e.g. the root syllable). /H, M, L/ is thus more compatible with languages like Fe’fe’-Bamileke, where words are short, with little morphology. Languages with shorter words often have more paradigmatic contrasts in general (more consonants, vowels--and tones). This may again be non-arbitrary, as the greater paradigmatic contrasts make up for the lost syllables of longer words. But it is not predictive.

\section*{Acknowledgements}
I would like to thank Nick Enfield for inviting me to the Ardennes workshop at which I also received several helpful responses. I am particularly indebted to Mark Dingemanse for his detailed review of an earlier version of this paper which has helped me clarify some of the points that I wanted to make.


{\sloppy
\printbibliography[heading=subbibliography,notkeyword=this]
}
\end{document}%%remove this line and move all lines below to localbibliography.bib
