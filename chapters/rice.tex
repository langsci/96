\documentclass[output=paper]{langsci/langscibook}
\ChapterDOI{10.5281/zenodo.573786}

\author{Keren Rice\affiliation{University of Toronto}}

\title{Dependencies in phonology: hierarchies and variation} 
%\abstract{Implicational scales are argued to play an important role in phonological theory, with statements such as ``if x, then y," where ‘x’ is considered to be more marked than ‘y’: the more marked is present only if the less marked is present. The question that I address in this note is how to account for the presence of cases in which ‘y’, the less marked, is absent, while ‘x’, the more marked is present. I focus on two types of situations. First, within a language, how do such implicational statements account for variation between a less and more marked sound? Second, considering different languages, why is it that, in neutralization positions, where the unmarked is expected, languages can differ in what can occur? I discuss variation and the limits on it, proposing that in the absence of a contrast, the hierarchies do not have a role to play: the outcome is affected not only by the phonetic and phonological biases in the scales, but also by historical and social factors.}



\begin{document}
 
\maketitle

\section{Introduction} \label{sec:intro}

\is{implicational scales} \is{markedness!hierarchies} \is{typology} Implicational scales, also often called markedness hierarchies, are proposed in linguistics to account for dependency relationships of the sort ``if x, then y," expressing typological generalizations. In general, a markedness hierarchy in phonology involves a family of related linguistic substantive features such as place of articulation and sonority; markedness hierarchies involving non-phonological features are also found, such as the well-known animacy and person hierarchies. In the equation if ``x, then y", x is considered to be more marked than y since the presence of y depends on the presence of x.\footnote{I discuss only one measure of markedness, namely implications. Note that many other factors have been identified with markedness. In general, unmarked is considered more basic, and is described with terms such as natural, normal, general, simple, frequent, optimal, predictable, ubiquitous, and acquired earlier; marked, on the other hand, is described with terms including less natural, less normal, specialized, complex, less frequent, less optimal, unpredictable, parochial, and acquired later. See, for instance, \citet{Hume2011} and \citet{Rice2007}. }
\is{implicational scales} \is{markedness!hierarchies} \is{typology} Implicational scales, also often called markedness hierarchies, are proposed in linguistics to account for dependency relationships of the sort ``if x, then y," expressing typological generalizations. In general, a markedness hierarchy in phonology involves a family of related linguistic substantive features such as place of articulation and sonority; markedness hierarchies involving non-phonological features are also found, such as the well-known animacy and person hierarchies. In the equation if ``x, then y", x is considered to be more marked than y since the presence of y depends on the presence of x.\footnote{I discuss only one measure of markedness, namely implications. Note that many other factors have been identified with markedness. In general, unmarked is considered more basic, and is described with terms such as natural, normal, general, simple, frequent, optimal, predictable, ubiquitous, and acquired earlier; marked, on the other hand, is described with terms including less natural, less normal, specialized, complex, less frequent, less optimal, unpredictable, parochial, and acquired later. See, for instance, \citet{Hume2011} and \citet{Rice2007}. }

\is{vowel height} There are numerous examples of such hierarchies in phonology. \citet{Beckman1997}, for instance, utilizes a vowel height markedness hierarchy to account for the presence of mid vowels in an inventory implying the presence of high and low vowels, but not vice versa (\citealt{Beckman1997}: 14, drawing on surveys of vowel inventories by \citealt{Crothers1978} and \citealt{Disner1984}), as in (1).

\ea%1
\label{ex:1}
    \ea  *Mid {\textgreater}{\textgreater} *High, *Low 
    \z
    \z


\is{optimality theory} This is to be read as follows: mid vowels are more marked than high vowels and low vowels. Hierarchies of this sort are designed to account for a variety of aspects of phonology including inventory structure and asymmetries in terms of processes such as neutralization and assimilation. While they have precedents in other theories, Optimality Theory makes particularly strong use of such hierarchies; see, for instance, \citet{Beckman1997}, \citet{Lombardi2002}, \citet{Hayes2004}, and \citet{DeLacy2006}, among others. The hierarchies are based on typological findings and expressed with substantive features involving phonetic categories. \is{typology}

\is{inventory structure} \is{markedness} In this chapter, I focus on dependencies as they relate to inventory structure and markedness. Perhaps the most extensive recent work on markedness and inventories is found in \citet{DeLacy2006}, working within an Optimality Theory framework. De Lacy makes very explicit claims about when unmarked surface forms are predicted, and I draw heavily on his work in the following discussion.

In discussion of diagnostics for markedness, \citeauthor{DeLacy2006} notes that inventory structure is a valid diagnostic ``to a very limited extent"(\citeyear[343]{DeLacy2006}). More particularly, he says that “If the presence of [$\alpha $] in a segmental surface inventory implies the presence of [$\beta $] but not vice versa, then there is some markedness hierarchy in which [$\beta $] is more marked than [$\alpha $].” He continues with a concrete example based on place of articulation, \is{place of articulation} for which he proposes that dorsal and labial places of articulation are more marked than coronal and glottal places of articulation: “if there is a dorsal and/or labial of a particular \is{manner of articulation} manner of articulation in a language, then there will also be a glottal and/or a coronal of the same manner of articulation (as long as no interfering manner-changing processes apply). Consequently, there must be one or more hierarchies in which dorsals and labials are more marked than coronals and glottals.” \citet[110]{DeLacy2006} further notes that in the absence of faithfulness constraints (constraints functioning to preserve input forms) and competing hierarchies, markedness is “decisive in selecting the output form,” known in the Optimality Theory literature as the emergence of the unmarked. The emergence of the unmarked refers to situations where a marked structure is generally allowed in a language, but is banned in particular contexts. The emergence of the unmarked is found in epenthesis, where the quality of an epenthetic segment is considered to be unmarked since there is no input correspondent, and in neutralization, as discussed below, among other contexts.

As mentioned above, in this chapter I examine hierarchies with respect to inventory structure, particularly addressing the claim that the presence of a more marked feature in a language implies the presence of a less marked one, determinable on universal grounds. I do this through two lenses. \is{variation} First I consider variation in the realization of a sound within a language, asking why it is that variation between a segment with a more marked \is{markedness} feature and one with a less marked feature on the same hierarchy should exist if hierarchies predict that presence of the more marked one implies the presence of the less marked one. Second I examine cross-linguistic aspects of place neutralization in coda position, asking why languages differ in possible places of articulation in a position where no contrasts exist, and where the presence of the least marked is predicted. \is{place neutralization}

I focus in particular on the place of articulation hierarchy, as in (2) \citep[e.g.,][]{DeLacy2006}. \is{place of articulation!hierarchy}

\ea%2
    \label{ex:2}
	  Dorsal {\textgreater}{\textgreater} Labial {\textgreater}{\textgreater} Coronal {\textgreater}{\textgreater} Glottal
    \z


According to this hierarchy, dorsals are the most marked consonants in terms of place of articulation, and glottals are the least marked. Thus, all other things being equal, one would expect that if there is a dorsal stop present in a language, there will also be a labial stop, and so on. Moreover, in the absence of a contrast, coronals or glottals should arise.

It is important to comment briefly on the notion of all other things being equal. While the place of articulation hierarchy is as in (2), de Lacy notes that both coronals and glottals can pattern as unmarked. He argues that this is due to the fact that while glottals are the least marked on the place of articulation hierarchy, they are more marked than other places of articulation on the sonority hierarchy: relations between features can be different depending upon the hierarchy at issue. It is thus important to examine features that are always in the same markedness relationship with one another; the place features Dorsal, Labial, and Coronal are assumed to be such features, and I focus on these places of articulation, leaving glottals aside. Thus I focus on situations where the only relevant hierarchy is the place of articulation hierarchy. Assuming this, there are very clear predictions: one would expect, both within and between languages, that under equivalent conditions, there would be uniformity. I begin by studying within language variation (\S \ref{sec:variation}) and then turn to between language variation (\S \ref{sec:word-final}).

\section{Variation within a language: place of articulation} \label{sec:variation}

\is{variation!intra-language} \is{place of articulation} Many languages are reported to exhibit variation in place of articulation in particular positions. For instance, in some languages there is no contrast in a particular position between coronal and dorsal stops or between coronal and velar nasals. Given the absence of a contrast, one would predict that the less marked place of articulation would be found. However, rather than the unmarked one occurring, in many languages both coronal and dorsal consonants of a particular manner of articulation are in variation with each other even though coronals are less marked than dorsals. In other cases there is no contrast between dorsal and uvular sounds at a manner of articulation and stops of these places of articulation are in variation even though dorsals are considered to be less marked than uvulars. A few examples of languages illustrating such variation are given in (3). In these cases, the variation is not controlled by linguistic factors; there may be social and other factors involved, but these are not mentioned in the literature.

\ea%3
    \label{ex:3}
   
    
\ea  coronal/dorsal variation
  \ili{San Carlos Apache} (\ili{Athabaskan}) \citep{DeReuse2006}

\-\hspace{0.5cm}[t]{\textasciitilde}[k] stem-finally 

\ili{Panare} (\ili{Cariban}): /n/ \citep{Payne2013}

\-\hspace{0.5cm}[n]{\textasciitilde}[ŋ] word-finally 

\ex
  dorsal/uvular variation 

\ili{Sentani} (\ili{Papuan}): /k/ \citep{Cowan1965}

\-\hspace{0.5cm}[k]{\textasciitilde}[q]{\textasciitilde}[x]

\ili{Qawasqar} (\ili{Alacalufan}) \citep{Maddieson2011}

\-\hspace{0.5cm}uvular{\textasciitilde}velar stop
\z
\z

As (3) shows, there may be variation in the realization of place of articulation within a language (see section 3 for some examples of variation of place of articulation involving labials). 

\is{markedness!hierarchies} \is{variation} This kind of variation is unexpected, given the type of fixed substantive markedness hierarchies discussed in \S \ref{sec:intro}. Note that while variation might follow as a result of conflicting hierarchies \citep[344]{DeLacy2006}, when all features save the varying one are controlled for, a solution to this problem grounded in conflicting hierarchies does not seem to be appropriate: as discussed earlier, there are no proposed hierarchies where coronal is more marked than dorsal, for instance, and dorsals are generally considered to be less marked than uvulars; in other words, there is no hierarchy where these are reversed. Recognizing this, \citet[341]{DeLacy2006} notes that “The markedness status of freely varying allophones is also unclear: underlyingly marked values do not only vary freely with less-marked ones,” and he further writes that “allophonic free variation should not be expected to show markedness effects” since it is due to phonological processes that may either “reduce markedness (e.g. neutralization)” or “inadvertently increase it (e.g. assimilation)" (\citeyear[342]{DeLacy2006}). In the languages given in (3), the variation is found either in a typical neutralization position, or appears to be free. Such variation gives pause, and I examine an alternative account to de Lacy’s in \S \ref{sec:what?}. 

\section{Cross-linguistic variation: word-final position (position of neutralization)} \label{sec:word-final}
 \largerpage[-1]
\is{variation!inter-language} \is{markedness!hierarchies} \is{neutralization} Important evidence for markedness hierarchies can be drawn from neutralization, as discussed in \citet{Trubetzkoy1969} and much subsequent work. See \citet{Battistella1990} for a review of literature on neutralization and Rice (2007, 2009) for more in-depth development of the ideas that are summarized in this chapter.

It is again instructive to consider de Lacy’s statements about neutralization as a diagnostic for markedness. De Lacy (\citeyear[342]{DeLacy2006}) recognizes the following aspect of neutralization as a relevant markedness diagnostic: “If /$\alpha $/ and /$\beta $/ undergo structurally conditioned neutralization to map to output [$\alpha $], then there is some markedness hierarchy in which [$\beta $] is more marked than [$\alpha $].” He further notes that not all neutralization presents valid diagnostics for markedness: “If /$\beta $/ undergoes neutralization but /$\alpha $/ does not, then it is not necessarily the case that there is a markedness hierarchy in which /$\beta $/ is more marked than /$\alpha $/” \citep[340]{DeLacy2006}. 

One can then look to neutralization positions for evidence for a markedness hierarchy, focusing on cases where there is neutralization between features of the same class (the valid instance noted by de Lacy). Word- and syllable-final positions are well-known sites of neutralization. For instance, neutralization of a laryngeal contrast to voiceless in these positions is very common. In addition, place of articulation neutralization can occur in these positions. \is{place neutralization} \is{place of articulation!hierarchy}Thus, given the place of articulation hierarchy in (2), one would expect to find neutralization to either coronal or glottal place of articulation; I again set aside glottal since it enters into the sonority hierarchy as well as the place of articulation hierarchy.

In the following discussion, I distinguish two types of neutralization, passive and active. Passive neutralization is a result of the lexicon: there are simply no lexical contrasts between features on some dimension in a particular position. For instance, with respect to place of articulation, only a single place of articulation is found in some position, with no evidence from alternations for active neutralization. Active neutralization is what the name implies: there is evidence that one place of articulation actively neutralizes to another. 

 
I begin with passive neutralization in word-final position, considering languages with a contrast between labials, coronals, and dorsals in their full inventory. I carried out a detailed survey of languages based on grammars and phonological descriptions, focusing on the places of articulation found in word-final position in stops and in nasals. A sampling of the results of this survey is provided in \tabref{tab:rice:absencestops} for stops and \tabref{tab:rice:absencenasals} for nasals in word-final position in languages where there is no contrast in place of articulation found in this position.

\newcolumntype{L}[1]{>{\raggedright\let\newline\\\arraybackslash\hspace{0pt}}m{#1}}

\begin{table}
    \caption{Absence of contrast in place of articulation word-finally: stops}
    \label{tab:rice:absencestops}

\begin{tabularx}{\textwidth}{lllL{0.8\textwidth}}
\lsptoprule
p & t & k & Languages\\
\midrule
\raisebox{0.6\normalbaselineskip}{x}&  &  & \ili{Nimboran} (\ili{Papuan}),
	  \ili{Basari} (\ili{Niger-Congo}), 
	  \ili{Sentani} (\ili{Papuan}), 
	  some \ili{Spanish} (\ili{Romance})\\
& x &  & \ili{Finnish} (\ili{Finno-Ugric}), 
	  \ili{Alawa} (Australia)\\
&  & x & \ili{Ecuador Quichua} (\ili{Quechuan}), 
	  \ili{Arekuna} (Carib)\\
\lspbottomrule
\end{tabularx}
\end{table}
    
\begin{table}
    \caption{Absence of contrast in place of articulation word-finally: nasals}
    \label{tab:rice:absencenasals}
    
\begin{tabularx}{\textwidth}[t]{lllL{0.8\textwidth}}
\lsptoprule

m & n & ŋ & Languages\\
\midrule
\raisebox{\normalbaselineskip}{x} &  &  & \ili{Sentani} (\ili{Papuan}), 
	  some \ili{Spanish} (\ili{Romance}), 
	  \ili{Kilivila} (\ili{Austronesian}), 
	  \ili{Mussau} (\ili{Austronesian}; [n] in names, borrowings; [ŋ] in one word)\\
& \raisebox{0.6\normalbaselineskip}{x} &  & \ili{Finnish} (\ili{Finno-Ugric}), 
	  \ili{Koyukon} (\ili{Athabaskan}), 
	  some \ili{Spanish} (\ili{Romance})\\
&  & \raisebox{0.6\normalbaselineskip}{x} & \ili{Japanese}, 
	  \ili{Selayarese} (\ili{Austronesian}), 
	  some \ili{Spanish} (\ili{Romance}), 
	  \ili{Macushi} (\ili{Cariban})\\
\lspbottomrule
\end{tabularx}
\end{table}


Many languages exhibit active neutralization to a single place of articulation in word- or syllable-final position. The expectations are clear: coronals (and glottals) are expected. Again I set aside glottals. Coronals indeed result from active neutralization in a number of languages including \ili{Saami} \citep[\ili{Uralic},][]{Odden2005} and \ili{Miya} \citep[\ili{Chadic},][]{Schuh1998}. However, labials and dorsals also occur as the sole place of articulation in neutralization positions. Examples of languages are given in (\ref{neutralization}); some languages are listed twice because variation is reported.

\ea \label{neutralization}
   neutralization to Labial: \ili{Manam} \citep[\ili{Austronesian},][]{Lichtenberk1983}, \ili{Miya} \citep[\ili{Afro-Asiatic},][]{Schuh1998}, \il{Spanish!Buenos Aires Spanish}Buenos Aires Spanish (\ili{Romance}, Smyth p.c.) %P.C contact needs full reference?

  neutralization to Dorsal: \ili{Manam} \citep[\ili{Austronesian},][]{Lichtenberk1983}, some \ili{Spanish} dialects (syllable-final), \ili{Carib} of Surinam (Carib; neutralization to [x] in syllable-final position, \citealt{Hoff1968}), \ili{Tlachichilko Tepehua} \citep[\ili{Totonacan}; neutralization to dorsal in syllable-final position,][]{Watters1980}
\z

One can conclude that, despite the wide range of evidence that is compatible with the place of articulation hierarchy (and other substantive hierarchies), in fact there are counterexamples where the unmarked does not emerge when it is expected. 

In the next section, I examine a possible reason for this: the fixed substantive universal hierarchies cannot provide insight into the non-contrastive kind of variation considered above, either within or between languages, because, in the absence of contrast, substance is not determinate \citep[see, for instance,][]{Rice2007,Rice2009,Hall2011}.

\section{What is going on?} \label{sec:what?}

\is{faithfulness constraints}  \is{markedness}The substantive generalizations in the hierarchies predict, as \citet{DeLacy2006} emphasizes, that, in the absence of faithfulness constraints (constraints that maintain input independent of its markedness) and competing hierarchies, markedness is “decisive in selecting the output form” \citep[110]{DeLacy2006}. In the types of cases discussed above, faithfulness is not at issue and, given that the outcomes under discussion share in all but place features, competing hierarchies do not offer insight as the places of articulation under consideration do not enter into alternative hierarchies. One can then ask why, despite the predictions of the hierarchy, such variation is found both language-internally and cross-linguistically. In this section I introduce another possibility, that, in the absence of an opposition, substantive hierarchies do not make predictions; rather phonetic naturalness and other factors are at play.

\is{markedness!semantic} \citet{Battistella1990}, in a detailed discussion of semantic markedness, provides interesting insight into the conditions under which it is relevant to talk about markedness. In particular, Battistella notes that marked elements “are characteristically specific and determinate in meaning.” Further, he continues, the opposed unmarked elements “are characteristically indeterminate” \citep[27]{Battistella1990}. He concludes that “whenever we have an opposition between two things, one of those things – the unmarked one – will be more broadly defined” \citep[4]{Battistella1990}.

\is{markedness} I draw two conclusions from \citet{Battistella1990}. First, unmarked elements are more general in interpretation than are marked elements, which have a more specific interpretation. This suggests, for instance, that the unmarked might show more phonetic variation than the marked. Second, and more relevant in a discussion of dependencies, given that markedness is defined with reference to oppositions, it is difficult to know how to understand markedness in the absence of an opposition, where there are simply not two (or more) elements to compare. Battistella focuses on the existence of an opposition between two (or more) features; under such a situation, one can be characterized as unmarked with respect to a particular hierarchy. What about when there is not an opposition? 

The variation within languages and the various possible outcomes of neutralization across languages lead us to a different conclusion than that predicted by the markedness hierarchies. Instead of assuming that, all other things being equal, markedness selects the output form, an alternative account is possible: all other things being equal, the substance of the output form is phonologically indeterminate in the absence of an opposition, or a contrast. I will call the first of these the emergence-of-the-unmarked approach and the second the absence-of-an-opposition approach.

\is{variation!intra-language} \is{variation!inter-language} The first approach, emergence-of-the-unmarked, predicts substantive uniformity cross-linguistically in the absence of competing hierarchies and faithfulness. The second approach, absence-of-an-opposition, predicts a certain amount of variability cross-linguistically (and language-internally as well). As discussed above, such variability can be captured by the emergence-of-the-unmarked approach through the establishment of different hierarchies where a particular feature is unmarked on one but not on another, with one or the other hierarchy privileged in different languages. However, as noted above, in terms of place of articulation, setting glottals aside, to my knowledge there are no proposals that make, for instance, coronal consonants unmarked on one hierarchy and marked on another, or labial consonants more marked than coronal consonants on one hierarchy but less marked on another. 

The absence-of-an-opposition approach predicts that either a coronal or a labial, for instance, could emerge in a position where there is no contrast. It is not a substantive markedness hierarchy that determines the outcome. Instead, any place of articulation is conceivably possible.

Given this latter approach, some important questions arise, and I briefly consider two of these. First, why have markedness hierarchies been proposed, with considerable empirical support? Another way of putting this is to ask why there are cross-linguistic biases. Second, if the unmarked truly is indeterminate in a universal sense, what factors are involved in determining the actual substance in a language?

\largerpage
\is{phonetic naturalness} \is{markedness!hierarchies} The answer to the first of these questions is reasonably straightforward: there are clear biases towards phonetic naturalness, represented in the markedness hierarchies by what is at the unmarked end of the hierarchy. For instance, \citet[39-40]{Maddieson1984} notes the following generalizations with regard to stops. The number following the generalization indicates the percentage or number of languages in the survey that obey the particular generalization.

\begin{itemize}
 \item All languages have stops. (100\%)
 \item If a language has only one stop series, that series is plain voiceless stops. (49/50 languages – 98.0\%)
 \item If a language has /p/ then it has /k/, and if it has /k/ then it has /*t/ (4 counterexamples in the UPSID sample; ‘*t’ signifies a dental or alveolar stop). 
\end{itemize}

Given these observations, one can make the following predictions.

\begin{itemize}
 \item Stops are expected to be less marked in manner than other obstruents.
 \item Plain voiceless stops are expected to be less marked than stops with other laryngeal features.
 \item Coronal stops are expected to be less marked than stops of other places of articulation.
\end{itemize}

Maddieson is clear that these are tendencies, or biases, as is well recognized in the literature. What then do we make of the counterexamples? 

I will very briefly note some possible contributing factors. First, articulatory and perceptual factors are important in establishing the widespread cross-linguis-tic uniformity, or biases, and these are well captured by the markedness hierarchies, accounting for the considerable cross-language convergence that we find. 

\is{diachrony} However, other factors are important as well. Diachronic factors can play a role, and in this case unexpected situations might arise \citep[see, for instance][]{Blevins2004}. For instance, \citet{Blust1984} attributes the presence of final /m/ and the absence of other word-final consonants in the \ili{Austronesian} language \ili{Mussau} to the loss of a vowel following this consonant (with frequent devoicing but not loss of final vowels following other consonants). 

Societal and social factors most likely are also important in shaping what is allowed in the absence of contrast \citep[see, for instance,][]{Guy2011}. \citet{Trudgill2011} identifies a number of societal factors that are involved in what he calls linguistic complexification, focusing on language size, networks, contact, stability, and communally-shared information. For instance, he notes that social isolation often contributes to the existence of both large and small inventories, unusual sound changes, and non-maximally dispersed vowel systems. One might imagine then that there might be a greater tendency to variation and less common outputs of neutralization in closely knit societies with relatively large amounts of shared equilibrium, where, Trudgill notes, less phonetic information is needed for successful communication. Research to establish whether such correlations do exist remains to be done.

\section{Conclusion}
\largerpage
It is very common to posit dependencies in the form of substantive hierarchies in linguistics. I have not addressed the overall status of such hierarchies in phonology, but have simply asked whether the hierarchies are determinate in the absence of a contrast. I have examined variation in place of articulation within a language and different outcomes of place of articulation neutralization between languages, and found that, all other things being equal, in fact the unmarked is not necessarily found. I conclude that, assuming that the evidence for substantive markedness hierarchies holds overall, they play a role only in the presence of contrasts; in the absence of an opposition, they are not determinate. The frequency of particular phonetic outcomes depends to a large degree on articulatory and perceptual factors, or phonetic naturalness, with diachronic and sociolinguistic factors also playing roles. It is important to understand when dependencies might indeed be a part of shaping a language, and when their existence masks a more nuanced situation.


{\sloppy
\printbibliography[heading=subbibliography,notkeyword=this]
}

\end{document}
