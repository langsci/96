\documentclass[output=paper]{langsci/langscibook}
\ChapterDOI{10.5281/zenodo.573785}

\author{Elena Lieven\affiliation{ESRC International Centre for Language and Communicative Development \newline(LuCiD)
\\
Division of Human Communication, Development and Hearing
\\
University of Manchester}}
\title{Is language development dependent on early communicative development?} 
%\abstract{This paper addresses the question of what has to be in place to allow language learning to ‘get off the ground’. \citet{Enfield2013} has argued that formed-up social cognition is a pre-requisite for successful communication and a fundamental tenet of the usage-based position on language learning is that children’s language development  is based on their early intention-reading skills \citep{Tomasello1999}. From this theoretical position, children learn language through understanding that people are communicating with them and they progressively learn the form-meaning mappings involved in these intentional interactions. This stands in contrast to other positions that argue for a more or less complete separation between learning the fundamental syntax of a language and using it to communicate \citep{Pesetsky1999}. I address this issue by exploring claims to dependencies from studies of children diagnosed with autism spectrum disorder (ASD). Despite the fact that typically developing children show an impressive range of social mindreading skills before they ever start to talk, children with ASD for which the defining characteristic is impaired social interaction, can range in language skills from being entirely mute to having highly developed language, thus potentially questioning the dependency between social cognition and language. Although there are a number of studies which follow children diagnosed with ASD, the combination of the fact that children are rarely diagnosed before they are three years old and that standardised tests are used, means that it is very difficult to get the rich developmental picture we would need to understand the processes underlying these children’s language development. In my conclusion, I raise the possibility that there may be more than one route into learning language and, indeed, more than one way of using it as an adult.}
\maketitle
\begin{document} 
\section{Introduction}

\is{Autism Spectrum Disorder} \is{shared intentionality} \is{language development} Children diagnosed with Autism Spectrum Disorder (ASD) show impairments in communication, social interaction and a restricted behavioural repertoire.  One influential hypothesis in the literature is that the understanding of other minds (i.e. that one’s interactants are communicating intentionally) is \textit{the} (or \textit{a}) necessary precondition to learning language.  Since, on the one hand, most children subsequently diagnosed with autism show disruption in measures of early intention reading and, on the other, some children diagnosed with autism learn to talk -- in some cases with real proficiency -- this seemingly challenges the above hypothesis (but see \citeauthor{Carpenter2000} \citedate{Carpenter2000}).  

Studies of later language development in autism have come to highly variable conclusions, some finding considerable differences with matched typically developing (TD) controls, others finding almost no differences in vocabulary or syntax though pragmatic skills may be impaired.  A recently published survey of language and communicative development in autism \citep{Arcuili2014} which covers many aspects from prelinguistic communication through to literacy, narrative, and conversational development shows this lack of agreement in the field for almost every aspect studied. In this chapter, I will first outline the claim that shared intentionality is a necessary foundation for language development before covering studies that have examined this in children who develop autism.  I will then look at the evidence for language impairments in autistic children.  \is{shared intentionality} 

\section{Shared intentionality as the precondition for language development}

\is{shared intentionality} There is pretty unanimous agreement that typically developing children show a qualitative change in interactive behaviour starting sometime around the last trimester of the first year.  Of course, this is preceded by other important developmental milestones: for instance, the onset of social smiling and the development of attachment-related behaviours.  Although termed the ``9-month revolution" by Tomasello and others, this overstates the abruptness of the shift in interactional behaviours, which show continuous development over this period.  The underlying theoretical construct is that of ``shared intentionality" -- a new world of shared intersubjectivity in which infants start to realise that others have intentions and that these can be related to their own intentions, i.e., that others are intentional agents like themselves.  The behavioural manifestations of this change in the understanding of other minds are ``triadic" interactions: interactions in which children involve their interactive partners in their own interests and actions and understand that the communicative behaviours of others are intentional.   The following behaviours are taken as evidence for this shift to ``intention reading": sharing joint attention to objects and knowing that you are doing so; showing objects to the other; using pointing to draw attention to events or objects; understanding what is new for the other; giving information to the other.  Tomasello characterises this as part of the human biological inheritance which allows for the cultural inheritance that we acquire through the specifically human behaviours of imitation, learning and teaching.  In turn, these form the basis for the ``cultural ratchet": the rapid rate of social and technological innovation and change in modern humans \citep[6]{Tomasello1999}. \is{joint attention} \is{biological inheritance} 

There does seem to be good evidence for a relatively universal developmental timetable for these early skills of shared intentionality \citep{Brown2011,Callaghan2011,Liszkowski2012,Lieven2013}, though as these studies also report, there are some differences resulting from the different cultural contexts (most importantly while \citeauthor{Callaghan2011} report language comprehension as starting at around 9-10 months in all the studied cultures, production is, on average, 3 months later in the non-technologically complex  cultures).  There also seems to be considerable consistency within a culture. A study by \citet{Carpenter1998} investigated the emergence of joint attentional skills in a group of 24 children in the USA aged between 9-15 months, as measured by 9 different tasks. They found that infants first shared attention, then started to follow the attention of the mother and finally started to direct attention.  There were also strong correlations between the emergence of each pair of skills and their sub-components: they emerged in close developmental synchrony and with a consistent ordering pattern.

\is{shared intentionality} \is{language development} Why should the development of shared intentionality be the necessary basis for language development?  The argument depends on understanding the importance of ``common ground" in all intentional communication.  The meaning of a communicative act can only be understood in a shared context. For instance \textit{The door is open} will be interpreted quite differently if someone is complaining about being cold rather than about being bored.   Therefore, the argument goes, infants will only be able to start to acquire language once they ``realise" that utterances addressed to them carry meaning based on shared common understandings.  Symbolic representations do not, therefore, exist cut off from their context but are always intersubjective (socially shared) and perspectival (they pick out a particular way of viewing a  phenomenon,  \citeauthor{Tomasello1999} 1999, \citeauthor{Levinson2006} 2006, \citeauthor{Enfield2013} 2013).  This potentially deals with the Quinian problem of how an infant can interpret the reference of an utterance, given the multitude of possibilities when the caretaker points and/or uses a word/sentence. To support this position, \citeauthor{Carpenter2000} ask why word learning takes off at 12-14 months and not much earlier given the enormous number of words that most infants hear during the first year of life.  Their answer is that the development of shared intentionality is crucial to providing the context in which word meaning can be interpreted, and therefore learned, and there is plenty of evidence that preverbal infants do, in fact, understand a good deal about what is given and new for another and can interpret other’s communication on this basis \citep{Tomasello2003attention,Moll2008}. This is supported by the many studies of typically developing children showing strong correlations between early joint attentional skills and vocabulary size (e.g. \citeauthor{Carpenter1998} \citeyear{Carpenter1998}). \is{preverbal infants}

\section{Studies of language development in autistic and ASD children}

\is{Autism Spectrum Disorder} A third to half of the children diagnosed with ASD never develop a functional language.  The rest do learn but with very varying degrees of sophistication \citep{Wetherby1992,Noens2006}.  The biggest problem in trying to understand these children’s language development is that different studies conflict in critically important ways.  There are a number of reasons for this.  The first is methodological: studies use different diagnostic criteria, different types of control groups and different methods of assessing children’s language and communicative development.  In the latter case, this is almost always done using standardised tests which do not give much insight into the underlying processes involved in developing language.  In addition, with the exception of the ``prodromal" studies mentioned below, since an autism diagnosis is rare before 3 years of age, the crucial early stages of breaking into language have not been available for study.  However there are some general conclusions that one can draw from this literature.  Children diagnosed with ASD \is{Autism Spectrum Disorder} \is{echolalia} usually show difficulties in communicative reciprocity and discourse management \citep{Anderson2009} and jargon echolalia is often present \citep{Roberts2014}. On standardised language tests,  children diagnosed with ASD are almost always behind compared to age-matched, TD controls.  However, if they are matched for mental age or vocabulary size, a number of studies find no difference in syntax or morphology. For instance, \citet{Brock2014} found that reading for words and sentences is largely predicted by degree of language impairment and level of oral language and \citet{Norbury2005} concludes that the oral comprehension of the children diagnosed with ASD in her study was predicted by their language skills and not the severity of their autism.  But how do these general findings for children aged 3;0 and above relate to the early development of shared intentionality? \is{shared intentionality}

\section{Prelinguistic communication in children who develop ASD}

\is{joint attention} There is a complex literature on the possible social interactional antecedents to language development in autism.  Different studies have focussed on particular aspects of early social interaction with Mutual Shared Attention, Joint Engagement, Response to Joint Attention and Initiation of Joint Attention held out as critical in different models with variable levels of evidence to support the claims. \citet{Sigman1999} followed 51 children with an autism diagnosis aged between 3-5 years of age when they were first recruited, into the mid-school years.  They found that joint attention behaviours by the children were strongly concurrently related to language skills.   Another study shows clear evidence of the involvement of child joint attention in predicting later communicative and language skills \citep{Siller2002}. As well, they also found that parental behaviours that were synchronised with their child’s focus of attention and ongoing activity  were associated with higher levels of joint attention in their children a year later and with better language outcomes 10 and 16 years later and this was independent of the child’s initial language age, IQ and joint attention skills.   In a separate study of a group of children who entered with a mean age of 16 months (and a standard deviation of 7 months),  the same authors \citep{Siller2008} found that, on the one hand, child characteristics on entry (Non-verbal IQ, language age as well as joint attention) were correlated and predicted language outcomes.  But, on the other hand, \textit{rate} of language growth was independently predicted by (a) children’s responsiveness to others’ bids for joint attention and (b) parents’ responsiveness to their children’s attention and activity during play and neither of these relations could be explained by initial variation in mental age or initial language abilities.  Thus there seems to be clear evidence that aspects of joint attention in children with ASD \is{Autism Spectrum Disorder} are implicated in subsequent language development and that parental success in achieving synchronous joint attention with their children is independently associated with more successful language outcomes.  However the fact remains that impaired joint attention is almost universally found in children with ASD and yet many do achieve competence in language at least to the level of using phrasal speech and sometimes to much more sophisticated language.  \is{joint attention}

A major development in the attempt to explore the developmental antecedents to autism comes from prodromal studies with the younger siblings of children already diagnosed with an autism spectrum disorder in which the probability of a sibling also developing the disorder is 20\% \citep{Ozonoff2011}.  This has led to a number of studies in which ``prodromal" children’s early communicative interaction is compared with that of low-risk children and then related to the subsequent outcome in terms of an ASD diagnosis \citep{Jones2014,Wan2013,Green2013}

The \citet{Wan2013} study which compared a prodromal high-risk group and a low risk group, used a global measure of the quality of mother-infant interaction at 8 and 14 months. The study showed that when compared to low-risk infants, at risk infants show significantly lower scores at 8 months than non at risk infants on global measures of the quality of parent-child interaction (PCI), \is{parent-child interaction} differences that at 14 months are increased and are associated with an autism outcome at 3 years of age. It should be emphasised that the authors consider that the lower measures of PCI quality are due to aspects of the infants’ behaviour (e.g. lack of eye contact) which arise from the infant’s condition, which then, in turn, disrupts the interaction between parent and child and thus the child’s functional social experience.  A targeted intervention study between 9-14 months succeeded in improving the quality of these interactions as well as suggesting a reduction of autism pre-symptoms at 14 month endpoint \citep{Green2013}. These improvements were sustained at 24 month follow up \citep{Green2015Parent}. At 14 months the non-significant trend in the data was for there to be, if anything, a slowing in \is{language acquisition} language acquisition – however by 24 months the treatment group showed a trend towards improved function, especially in receptive language development.  There was however no equivalent effect on ``structural" language development, suggesting a possible relative dissociation in this context between the quality of PCI \is{parent-child interaction} and attention on the one hand and syntax growth on the other. This suggests that while being able to respond to joint attention initiatives and caregivers’ ability to synchronise communication with the child are facilitatory in learning language,  they may well not be essential, potentially contra to a strong version of the Tomasello hypothesis.

\largerpage
\section{Implications}

\largerpage
There are, of course, many interpretations of what it means to learn language.  Minimally, I mean the ability to produce and understand what is said in some relation to actions and events, at least one’s own, and to be able to adapt one’s utterances to different situations with at least some ability to go beyond reproducing utterances learned by rote.  

\is{Autism Spectrum Disorder} \is{Williams syndrome} The suggestion that the development of language within autism progresses in rather a different way to that of typical language development has often been raised but the evidence currently is not sufficient to decide whether this is the case nor to understand the mechanisms which might underpin any such differences. \citeauthor{Karmiloff-Smith2006}, in her studies of children with Williams syndrome (\citedate{Karmiloff-Smith2006}), has suggested that these children’s facility with language (relative to very low levels of cognitive ability) might represent a different learning route.  Can we suggest the same thing for those children with ASD who learn language?  How might children who are more or less impaired on early intention reading skills learn language?  Clearly there is an innate basis to the learning of language but this leaves open a very wide range of possibilities.  First, language learning might actually be independent of the communicative basis with which language is used.  \is{biological inheritance} The best known version of this position argues for an innate set of specifically linguistic modules, one of which is Universal Grammar \is{universal grammar} (others that have been proposed are for phonology and semantics).  In this approach, communication may be largely what language is used for but this has nothing to do with how phonology, semantics and syntax develop.  This has been argued very strongly within the Generativist tradition but has recently met strong challenges from a constructivist, usage-based approach (see \citeauthor{Ambridge2011} \citedate{Ambridge2011}, \citeauthor{Ambridge2014} \citedate{Ambridge2014}). In terms of autism, the immense range of language outcomes seems to challenge the idea of an encapsulated syntactic module, in that children with ASD do not show an ``all-or-nothing" profile for syntax or, for that matter, any other aspect of language.  \is{generative linguistics} \is{autonomous syntax}

An alternative possibility is that since language learning is underpinned by a range of cognitive skills,  if some or all of these are relatively intact, structural language can be learned though its use may be pragmatically impaired.  For instance, there are word learning studies that suggest that attentional mechanisms and physical context information are sufficient for at least some word learning (Samuelson \& Smith 1998).  Once children can isolate some words (e.g. own name) this appears to facilitate learning (segmentation) of other words \citep{Fernald2006,DePaolis2014}. Both are potential non-social routes into language that have some empirical support.  

\is{statistical learning abilities} \is{echolalia} Minimally, infants need to be able to select relevant information, maintain focus/vigilance and move on or unstick from the current focus.  Other skills would involve strong statistical learning abilities, an intact working memory and rapid temporal order processing.   We know that many autistic children are echolalic, which suggests a good ability to retain short-term phonological information.  This is clearly not enough because many echolalic children never develop an innovative ability with language. It is also important to note that there is a variety of definitions of imitation, some of which are much more dependent on the imitator’s ability to ``mind-read" the goals of the imitated action (e.g. \citeauthor{Over2013} \citedate{Over2013}). However if the ability to learn from the statistical distribution of the words and inflections that infants hear in the language around them is also present, an enhanced imitative skill might provide a partial route into the learning of language structure.  A second pre-requisite might be the ability to ``parse" events and objects in the world.  This requires, first, the primate-wide abilities to cognitively represent spaces, objects and conspecifics and relational categories as well as the arguably more human cognitive capacities of categorisation, analogy and abstraction.  But all of this would require intact attentional skills.  The suggestion that some ASD children show abnormal attentional behaviour in infancy (faster to disengage from faces but also difficulties in disengaging from other stimuli \citep{Gliga2014} might be a factor in inhibiting this ability to relate what they hear to what they see.  For instance \citet{Ibbotson2014} showed that when mothers use the \ili{English} progressive this is significantly more likely to overlap with an ongoing event than is the case when the same verb is used with other temporal/aspectual marking.  If a child has a problem with rapidly shifting attention, they might well fail to pick up this form-meaning correlation with ``upstream" consequences for learning.   

\largerpage
These are just a few brief indications about how we might go about addressing this important issue.  By putting together findings of particular early impairments from the autism literature with a detailed analysis of how these might impact on the learning of language we could start to explore the possibility of different routes into more or less successful language learning.  This would also contribute to understanding the many other factors  involved in the learning of language by neuro-typical children and allow us to develop more nuanced theories which  attempt to integrate these factors  with an understanding of how early social cognition does and does not contribute to different aspects of language development.

A longitudinal prodromal study of the naturalistic communicative and linguistic behaviour of children at risk of an autism diagnosis which relates in depth assessment of language and pragmatic skills to antecedent variables will represent a significant contribution to our understanding of language development within the context of autism. We hope to undertake a study of this kind in the near future.  

{\sloppy
\printbibliography[heading=subbibliography,notkeyword=this]
}
\end{document} 
